\documentclass[]{tufte-handout}

% ams
\usepackage{amssymb,amsmath}

\usepackage{ifxetex,ifluatex}
\usepackage{fixltx2e} % provides \textsubscript
\ifnum 0\ifxetex 1\fi\ifluatex 1\fi=0 % if pdftex
  \usepackage[T1]{fontenc}
  \usepackage[utf8]{inputenc}
\else % if luatex or xelatex
  \makeatletter
  \@ifpackageloaded{fontspec}{}{\usepackage{fontspec}}
  \makeatother
  \defaultfontfeatures{Ligatures=TeX,Scale=MatchLowercase}
  \makeatletter
  \@ifpackageloaded{soul}{
     \renewcommand\allcapsspacing[1]{{\addfontfeature{LetterSpace=15}#1}}
     \renewcommand\smallcapsspacing[1]{{\addfontfeature{LetterSpace=10}#1}}
   }{}
  \makeatother

\fi

% graphix
\usepackage{graphicx}
\setkeys{Gin}{width=\linewidth,totalheight=\textheight,keepaspectratio}

% booktabs
\usepackage{booktabs}

% url
\usepackage{url}

% hyperref
\usepackage{hyperref}

% units.
\usepackage{units}


\setcounter{secnumdepth}{-1}

% citations
\usepackage{natbib}
\bibliographystyle{plainnat}


% pandoc syntax highlighting

% table with pandoc

% multiplecol
\usepackage{multicol}

% strikeout
\usepackage[normalem]{ulem}

% morefloats
\usepackage{morefloats}


% tightlist macro required by pandoc >= 1.14
\providecommand{\tightlist}{%
  \setlength{\itemsep}{0pt}\setlength{\parskip}{0pt}}

% title / author / date
\author{Martin Roberts}
\date{2022-05-13}


\begin{document}





\hypertarget{comm-7018-social-media-theory}{%
\subsection{COMM 7018: SOCIAL MEDIA
THEORY}\label{comm-7018-social-media-theory}}

Fitchburg State University\\
Communications Media Department\\
MS, Applied Communication, Social Media Concentration\\
GCE Online-Accelerated

7 weeks, Monday 16 May -- Monday 4 July 2022\\
Instructor: Dr.~Martin Roberts\\
Email:
\href{mailto:mrober40@fitchburgstate.edu}{\nolinkurl{mrober40@fitchburgstate.edu}}

\begin{center}\rule{0.5\linewidth}{0.5pt}\end{center}

\hypertarget{overview}{%
\subsubsection{OVERVIEW}\label{overview}}

The term \textbf{social media} is commonly understood as referring to
corporate-owned, advertising-funded communication \textbf{platforms}
based on user-generated content: YouTube, Instagram, Facebook, Twitter,
Twitch, Discord, TikTok. It can also be defined more broadly, however,
as a set of networked, technologically-mediated \textbf{practices} of
communication, structured by economic and political forces that both
inflect and are inflected by social and cultural identities. These
platforms, the social practices that they enable, and the relationship
between the two are the objects of \textbf{social media theory}. But
what does it mean to \textbf{theorize} social media? Why do we need
social media theory at all?

To theorize something involves a number of processes:

\begin{itemize}
\tightlist
\item
  first, how do we define the phenomenon or object of study itself? How
  does it differ from previous or other related phenomena?
\item
  how are we to account for it? Why did it happen/is it happening now
  rather than at some other time? What are its conditions of
  possibility?
\item
  what is its relation to larger areas of society? What are its
  implications for those areas?
\item
  how are we to evaluate it, in terms of its implications (political,
  economic, social, ethical, legal, environmental, aesthetic)? What are
  its possibilities and limits, its progressive and oppressive aspects?
  How can we change it for the better?
\end{itemize}

These processes involve developing analytical frameworks or models
comprising concepts that are useful for identifying and analyzing key
aspects of and issues raised by the phenomenon/object in question. These
frameworks and concepts typically draw from existing ones in different
fields of study, but often involve the proposal of new frameworks and
concepts specific to the field in question.

We begin by defining the object itself: what exactly do we mean by the
term \textbf{social media}? Haven't \textbf{media} - whether in the
ancient or modern sense - always already been \textbf{social}? How do
\textbf{social media} platforms differ from \textbf{social networking}
platforms? What is a \textbf{platform} anyway, and how does it differ
from a \textbf{medium}?

We then consider each week some of the key dimensions of social media
communication:

\begin{itemize}
\tightlist
\item
  the dimension of \textbf{sharing} that is arguably the central
  practice of social media;
\item
  the \textbf{affective} or emotionally-expresssive dimension of social
  media, i.e.~their role in articulating and sharing \textbf{structures
  of feeling};
\item
  the role of \textbf{language} in social media and its transformation;
  the concept of the \textbf{attention economy} that is the economic
  basis for commercial social media platforms;
\item
  the notion of social media as a complex \textbf{ecosystem} of
  interdependent practices, as well as proposals for addressing toxic
  elements within this ecosystem;
\item
  finally, we consider the possible \textbf{futures} of social media and
  both their potentially progressive and harmful implications as a site
  of struggle between competing forces and interests in the century
  ahead.
\end{itemize}

\hypertarget{objectives}{%
\subsubsection{OBJECTIVES}\label{objectives}}

By the end of the course, students will be able to:

\begin{itemize}
\tightlist
\item
  analyze technologies past, present, and imagined
\item
  describe the ways in which technologies shape our world the ways in
  which we shape those technologies
\item
  explain how social media is a result of the intersection between
  technologies and existing human communication dynamics
\item
  discuss how theory of technology and social media can improve the
  vocational outlook of a student
\item
  play a productive role in and even facilitate conversations that tease
  out the relationships between values and technology.\\
\item
  through the skills you will refine in writing your research papers,
  clearly explain how a specific technology shapes the social world that
  we live in.
\end{itemize}

\begin{center}\rule{0.5\linewidth}{0.5pt}\end{center}

\hypertarget{course-readings}{%
\subsubsection{COURSE READINGS}\label{course-readings}}

\textbf{Required}

\begin{itemize}
\tightlist
\item
  Nicholas A. John, \emph{The Age of Sharing}. Cambridge: Polity Press,
  2016.\\
\item
  Gretchen McCulloch, \emph{Because Internet: Understanding the New
  Rules of Language}. Riverhead Books, 2019.
\end{itemize}

\textbf{Recommended}

\begin{itemize}
\tightlist
\item
  Tim Hwang, \emph{Subprime Attention Crisis: Advertising and the Time
  Bomb at the Heart of the Internet}. FSG Originals, 2020.
\item
  Whitney Phillips and Ryan M. Milner, \emph{You Are Here: A Field Guide
  for Navigating Polarized Speech, Conspiracy Theories, and Our Polluted
  Media Landscape}. Cambridge, MA: MIT Press, 2021.\\
\item
  Joanna Zylinska, ed.,
  \href{https://kclpure.kcl.ac.uk/portal/en/publications/the-future-of-media(ed3f1ada-a46e-4c74-ac5b-9cd59000a732).html}{\emph{The
  Future of Media}}. London: Goldsmiths Press, 2022.
  \href{https://kclpure.kcl.ac.uk/portal/files/171061098/Future_of_Media_Open_Access.pdf}{Open
  Access}
\end{itemize}

\textbf{Other sources}\\
These sources are either freely available online or will be posted as
PDFs on Blackboard.

\begin{itemize}
\tightlist
\item
  Matthew Brennan, \emph{Attention Factory: The Story of TikTok and
  China's ByteDance}, 2020.
\item
  Yves Citton, \emph{The Ecology of Attention}. Cambridge: Polity Press,
  2017 {[}French edition: \emph{Pour une écologie de l'attention},
  Éditions du Seuil, 2014{]}\\
\item
  Daniel Miller, \emph{How the World Changed Social Media}. London: UCL
  Press, 2016.\\
\item
  Tony D. Sampson, Stephen Maddison, Darren Ellis, \emph{Affect and
  Social Media: Emotion, Media, Anxiety and Contagion}. London and New
  York: Rowman \& Littlefield, 2018.\\
\item
  José van Dijck, \emph{The Culture of Connectivity: A Critical History
  of Social Media}. Oxford: Oxford University Press, 2013.
\end{itemize}

\begin{center}\rule{0.5\linewidth}{0.5pt}\end{center}

\hypertarget{reading-schedule}{%
\subsubsection{READING SCHEDULE}\label{reading-schedule}}

\textbf{Week 1} M 05/16

Topic: Defining

\begin{itemize}
\tightlist
\item
  Daniel Miller, \emph{How the World Changed Social Media}: ``What is
  social media?,'' ``Academic studies of social media'' (chs.~1-2).
\end{itemize}

\begin{center}\rule{0.5\linewidth}{0.5pt}\end{center}

\textbf{Week 2} M 05/23

Topic: Speaking/Writing

\begin{itemize}
\tightlist
\item
  Gretchen McCulloch, \emph{Because Internet}: ``Informal Writing,''
  ``Typographical Tone of Voice,'' ``Emoji and Other Internet Gestures''
  (chs.~1, 4, 5)\\
\item
  \href{https://snowclones.org/about/}{The Snowclones Database}
\end{itemize}

\begin{center}\rule{0.5\linewidth}{0.5pt}\end{center}

\textbf{Week 3} M 05/30

Topic: Sharing

\begin{itemize}
\tightlist
\item
  José van Dijck, \emph{The Culture of Connectivity}: ``Facebook and the
  Imperative of Sharing'' (ch.~3)\\
\item
  Nicholas John, \emph{The Age of Sharing}: ``Introduction,'' ``Sharing
  and the Internet'' (ch.~3)
\end{itemize}

\begin{center}\rule{0.5\linewidth}{0.5pt}\end{center}

\textbf{Week 4} M 06/06

Topic: Feeling

\begin{itemize}
\tightlist
\item
  Nicholas John, \emph{The Age of Sharing}: ``Sharing our Feelings''
  (ch.~3)\\
\item
  Tony D. Sampson, Stephen Maddison, Darren Ellis, \emph{Affect and
  Social Media: Emotion, Media, Anxiety and Contagion}:
\end{itemize}

\begin{center}\rule{0.5\linewidth}{0.5pt}\end{center}

\textbf{Week 5} M 06/13

Topic: Attention

\begin{itemize}
\tightlist
\item
  Yves Citton, \emph{The Ecology of Attention}: ``Introduction: From
  Attention Economy to Attention Ecology''\\
\item
  Tim Hwang, \emph{Subprime Attention Crisis}: ``Introduction,'' ``The
  Plumbing,'' ``Subprime Attention'' (chs.~1, 4)\\
\item
  See also: Brennan, \emph{Attention Factory}, chs.~tba
\end{itemize}

\begin{center}\rule{0.5\linewidth}{0.5pt}\end{center}

\textbf{Week 6} M 06/20

Topic: Detoxing

\begin{itemize}
\tightlist
\item
  Whitney Phillips and Ryan M. Milner, \emph{You Are Here}:
  ``Cultivating Ecological Literacy'' (ch.~5)
\end{itemize}

\begin{center}\rule{0.5\linewidth}{0.5pt}\end{center}

\textbf{Week 7} M 06/27

Topic: The Future

Joanna Zylinska, ed., \emph{The Future of Media}:

\begin{itemize}
\tightlist
\item
  The Future of Social Media: Milly Williamson, ``The Celebrity Selfie:
  Gender, Race, and `New' Old Ways of Seeing'' (ch.~7)
\item
  The Future of Activism: Sue Clayton, ``How Smartphones and Digital
  Apps are Transforming Activist Movements'' (ch.~14).
\end{itemize}

\begin{center}\rule{0.5\linewidth}{0.5pt}\end{center}

\hypertarget{workflow-expectations}{%
\subsubsection{WORKFLOW \& EXPECTATIONS}\label{workflow-expectations}}

The above schedule is provisional; any changes to reading materials
and/or assignment deadlines will be announced on Blackboard.

You are expected to complete reading and and discussion board
assignments for each week the end of the day on Friday; discussion board
question posts are due Friday evening on their the weekly discussion
boards.

\hypertarget{rules-grading-policy}{%
\subsubsection{RULES \& GRADING POLICY}\label{rules-grading-policy}}

\textbf{Responsibilities}\\
You are required to use Blackboard during class. Course materials,
reading assignments, due dates and outlines are posted on Blackboard and
updated weekly: this means that assigned readings, assignments and due
dates may change. You are responsible for reading any course
announcements that are posted on Blackboard.

This class is a safe space from racist, sexist, or homophobic speech. If
you do use hate speech in any forum associated with the class you will
be subject to removal from class and possibly University disciplinary
action. Keep it clean, polite and respectful on class time.

\hypertarget{assignments-evaluation}{%
\subsubsection{ASSIGNMENTS \& EVALUATION}\label{assignments-evaluation}}

Grades are based on:

\begin{itemize}
\tightlist
\item
  Midterm project description: 25\%
\item
  Term project and presentation: 50\% (various components across the
  semester - see below for details)
\item
  Engagement and discussion board participation: 25\%
\end{itemize}

Midterm Project Description: 25\%

\textbf{Weekly Assignments}\\
By the end of the day on Friday of each week, you will have completed
one of the following:

\begin{itemize}
\tightlist
\item
  posted a short comment on and at least one question about the readings
  on the discussion board for the week
\item
  submitted at least one response to at least one classmate on their
  question
\end{itemize}

It's very important to keep up with and show that you are thinking about
the readings and your responses to them.

On \textbf{Monday} of each week, I will make an introductory post called
an Agenda that introduces and contextualizes the reading assignments for
the week, and identifies key themes, concepts, and/or issues to look out
for as you read. Be sure to read the Monday agenda post before beginning
the readings, and aim to complete them by around mid-week so that you
can post your response and/or questions.

\textbf{Term Project}\\
In week 2 you will submit a one-page project proposal, using the
proposal template I'll provide on Blackboard, on a topic relevant to the
course that is of interest to you. This is worth 5\% of your final
grade.

You're required to discuss your proposal with me before submitting it. I
will provide suggestions for research and development, along with
helping you refine your topic and project deliverable. A discussion
board will be available for you to submit ideas and discuss this with us
so all of us can talk about your idea.

Deliverables for the term project may include research papers, fieldwork
reports, media creations, app/software design, or other things in
consultation with me.

In week 4 you will submit a project progress report. This will focus on
what you've done to meet the goals you set out in your proposal and what
you need to do to complete successfully. This will be worth 5\% of your
final grade.

In week 7 you will deliver an audiovisual presentation on your project.
The presentation should address what your project was about, why you
chose it, the significance of the topic itself, how you went about
investigating it, and what you concluded. This will be worth 5\% of your
final grade.

To do this in PowerPoint: There should be a menu called ``Slide Show.''
On this menu, there's a choice called ``Slide transition (or just
Transitions) . . .'' When you select it, you'll see options for
advancing the slide. Clear the check box beside ``On Mouse Click,'' and
put (for example) ``20″ in the box: ``Automatically after seconds.''

Final presentations are hard. Rehearse, and then rehearse a lot more. I
will book a time for rehearsals the week before presentation day.
Attendance will be optional but strongly encouraged.

In week 7 you will also submit your paper. This will be worth 20\% of
your final grade.

\textbf{Discussion board questions}

For the weeks specified on the schedule (pay careful attention!):

\begin{itemize}
\tightlist
\item
  by the end of the day on Friday each week post one substantive
  question or comment related to the reading assignments for that week.
  You should also submit at least one comment in response to posts by
  other students.
\end{itemize}

The pupose of the question assignment is:

\begin{itemize}
\tightlist
\item
  to identify subjects/issues that are of interest to the entire group
  are there particular points or issues that consistently draw the
  class's attention?
\item
  to highlight aspects people might not have understood, or would like
  to see expressed more clearly during our class sessions.
\item
  to help us to get to know \emph{you} better: are you the person who
  always asks about economics, or politics, or pop culture?
\end{itemize}

\textbf{Reading questions and responses}\\
A \emph{weak} question is superficial, like a grade-school pop quiz
question: ``How does the author define X?'' ``How many copies of
\emph{Halo Reach} were sold last year?'' Weak questions have answers
already in the material or are stock questions that potentially apply to
anything: ``Isn't this bad for kids?''

A \emph{good} question calls for explanation or clarification: ``Does
Huizinga mean that games played for money aren't games at all, or that
they somehow stop being games when people get paid?''

An \emph{excellent} question link the reading to something else: ``How
does Huizinga's attitude towards being paid to play relate to the role
of amateurism in the Olympics and European aristocratic attitudes
towards labor?''

A sentence or two is fine. You won't get more credit for more words:
concision and clarity are good things.

Below are some examples of questions from a previous class (on games
studies) that represent ``good'' and ``excellent'' questions.

\emph{Example \#1:}\\
``I want to primarily focus on''Bow, N****``. How has the internet and
by extension online gaming allowed for individuals to express racial
views that are considered socially unacceptable? Do these people truly
feel this way or is there some sort of stimulus that promotes this usage
online?''

\emph{Example \#2}\\
``I'm talking about Alter Ego. The more options and choices that were
presented, the more unrealistic portions of the game seemed. At what
point do we sacrifice realism in favor of control over the story? And at
what point do we sacrifice control for the sake of realism? In my
experience some of the best entertainment has come from finding the
proper balance of these two, be it in a book, video game, or other form
of media It is this choice that leads to the complexity and realism in
life that I don't think video games are able to capture in its
complexity and nuances.

\begin{center}\rule{0.5\linewidth}{0.5pt}\end{center}

\hypertarget{gce-learning-competencies}{%
\subsubsection{GCE LEARNING
COMPETENCIES}\label{gce-learning-competencies}}

\textbf{Critical Thinking \& Problem Solving}\\
The exam and term project formats are designed to confront students with
a case drawn from current events in order to demonstrate their ability
to apply theories, diagnose problems, and suggest solutions, both in a
real-time, in-class exercise and over the course of the term in a more
deliberative manner. In-class discussion focuses on students' developing
and defending personal views on key issues based upon evidence and
theoretical frames.

\textbf{Communication}\\
Students are encouraged to develop strengths in both written and oral
communication, both deliberative and spontaneous, while the grading
rubric recognizes that students have different strengths and comfort
levels across media of expression. The format of the final presentation
places a premium on conciseness, mastery of the material to be
presented, and audience impact.

\textbf{Ethical Decision Making}\\
This course examines the ethical underpinnings of technology design and
use. It actively encourages students to discern and critique the ethical
components of technology and to assert their own ethical values in
technological analysis. The exams in particular call upon students to
articulate an ethical and theoretical underpinning for their diagnoses
of current problems and advocacy of solutions to those problems.

\textbf{Global Awareness}\\
This course addresses the impact of technologies at the global scale,
introduces concepts of global technological governance, and provides a
range of readings and guest speakers from outside the US.

\textbf{Civic Engagement}\\
This course does not call upon students to demonstrate civic engagement.
Rather, it provides tools of critical thinking, ability to recognize
global and local impacts, and the articulation of personal political
lenses based upon theoretical frameworks and data analysis, in order to
empower students with the tools for informed and effective engagement
with the impacts of technologies at local, national, and global levels.

\href{http://creativecommons.org/licenses/by-nc/4.0/}{This work is
licensed under a Creative Commons Attribution-NonCommercial 4.0
International License}

\hypertarget{bibliography}{%
\subsubsection{BIBLIOGRAPHY}\label{bibliography}}

\[A\] = Audible.com audiobook

\[A\] danah boyd, \emph{It's Complicated: The Social Lives of Networked
Teens} (New Haven: Yale University Press, 2014).

\[A\] Finn Brunton and Helen Nissenbaum, \emph{Obfuscation: A User's
Guide for Privacy and Protest} (Cambridge: MIT Press, 2016).

\[A\] Gabriella Coleman, \emph{Hacker, Hoaxer, Whistleblower, Spy: The
Many Faces of Anonymous} (London and New York: Verso, 2014).

David Craig, Jian Lin, and Stuart Cunningham, \emph{Wanghong as Social
Media Entertainment in China}. Palgrave Studies in Globalization,
Culture and Society (London: Palgrave, 2021).

Stuart Cunningham and David Craig, \emph{Social Media Entertainment: The
New Intersection of Hollywood and Silicon Valley} (New York: NYU Press,
2021).

---, eds.~\emph{Creator Culture: An Introduction to Global Social Media
Entertainment} (New York: New York University Press, 2021).

Eric Gordon and Adriana de Souza e Silva, \emph{Net Locality: Why
Location Matters in a Networked World} (Chichester, West Sussex, UK:
Wiley-Blackwell, 2011).

Byung-Chul Han, \emph{In The Swarm: Digital Prospects} (Cambridge: MIT
Press, 2017).

Sarah J. Jackson, Moya Bailey, et al., \emph{\#HashtagActivism: Networks
of Race and Gender Justice} (Cambridge: MIT Press, 2020).

Lori Kido Lopez, \emph{Race and Media: Critical Approaches} (New York:
New York University Press, 2020).

Lev Manovich, \emph{Instagram and Contemporary Image} (2015-17).

--- \emph{Cultural Analytics} (Cambridge: MIT Press, 2020).

\[A\] Gretchen McCulloch, \emph{Because Internet: Understanding the New
Rules of Language} (New York: Riverhead Books, 2019).

\[A\] Angela Nagle, \emph{Kill All Normies: Online Culture Wars From
4Chan and Tumblr to Trump and the Alt-Right} (Alresford, Hampshire, UK:
Zero Books, 2017).

Whitney Phillips, \emph{This Is Why We Can't Have Nice Things: Mapping
the Relationship between Online Trolling and Mainstream Culture}
(Cambridge: MIT Press, 2015).

\[A\] Whitney Phillips and Ryan M. Milner, \emph{You Are Here: A Field
Guide for Navigating Polarized Speech, Conspiracy Theories, and Our
Polluted Media Landscape} (Cambridge: MIT Press, 2021).

\[A\] Zoë Quinn, \emph{Crash Override: How Gamergate (Nearly) Destroyed
My Life, and How We Can Win the Fight Against Online Hate} (New York:
PublicAffairs, 2017).

\[A\] Jon Ronson, \emph{So You've Been Publicly Shamed} ( New York:
Riverhead Books, 2015).

\[A\] Kai Strittmatter, \emph{We Have Been Harmonized: Life in China's
Surveillance State} (New York: HarperCollins, 2020).

\[A\] Zeynep Tufekci, \emph{Twitter and Tear Gas: The Power and
Fragility of Networked Protest} (New Haven: Yale University Press,
2017).

\[A\] Siva Vaidyanathan, \emph{Antisocial Media: How Facebook
Disconnects Us and Undermines Democracy} (Oxford: Oxford University
Press, 2018).

\begin{center}\rule{0.5\linewidth}{0.5pt}\end{center}

\hypertarget{late-policy}{%
\subsubsection{LATE POLICY}\label{late-policy}}

Assignments that are late will lose 1/2 of a grade per day, beginning at
the end of class and including weekends and holidays. This means that a
paper, which would have received an A if it was on time, will receive a
B+ the next day, B- for two days late, and so on. Time management,
preparation for our meetings, and timely submission of your work
comprise a significant dimension of your professionalism. As such, your
work must be completed by the beginning of class on the day it is due.
If you have a serious problem that makes punctual submission impossible,
you must discuss this matter with me before the due date so that we can
make alternative arrangements. Because you are given plenty of time to
complete your work, and major due dates are given to you well in
advance, last minute problems should not preclude handing in assignments
on time.

\begin{center}\rule{0.5\linewidth}{0.5pt}\end{center}

\hypertarget{mandatory-reporter}{%
\subsubsection{MANDATORY REPORTER}\label{mandatory-reporter}}

Fitchburg State University is committed to providing a safe learning
environment for all students that is free of all forms of discrimination
and harassment. Please be aware all FSU faculty members are ``mandatory
reporters,'' which means that if you tell me about a situation involving
sexual harassment, sexual assault, dating violence, domestic violence,
or stalking, I am legally required to share that information with the
Title IX Coordinator. If you or someone you know has been impacted by
sexual harassment, sexual assault, dating or domestic violence, or
stalking, FSU has staff members trained to support you. If you or
someone you know has been impacted by sexual harassment, sexual assault,
dating or domestic violence, or stalking, please visit
\url{http://fitchburgstate.edu/titleix} to access information about
university support and resources.

\begin{center}\rule{0.5\linewidth}{0.5pt}\end{center}

\hypertarget{health}{%
\subsubsection{HEALTH}\label{health}}

\href{http://www.google.com/url?q=http\%3A\%2F\%2Fwww.fitchburgstate.edu\%2Foffices-services-directory\%2Fhealth-services\%2F\&sa=D\&sntz=1\&usg=AFQjCNEw5V0i0hL5DVO5b43gejNNaAt4ig}{Health
Services}

Hours: Monday-Friday 8:30AM-5PM Location: Ground Level of Russell Towers
(across from the entrance of Holmes Dining Hall) Phone: (978)
665-3643/3894

\href{http://www.google.com/url?q=http\%3A\%2F\%2Fwww.fitchburgstate.edu\%2Foffices-services-directory\%2Fcounseling-services\%2F\&sa=D\&sntz=1\&usg=AFQjCNEYiS4EmSvWerpp2bKr5lTpouPuqQ}{Counseling
Services}

The Counseling Services Office offers a range of services including
individual, couples and group counseling, crisis intervention,
psychoeducational programming, outreach ALTERNATIVE ECOSYSTEMSs, and
community referrals. Counseling services are confidential and are
offered at no charge to all enrolled students. Staff at Counseling
Services are also available for consultation to faculty, staff and
students. Counseling Services is located in the Hammond, 3rd Floor, Room
317.

\href{http://www.google.com/url?q=http\%3A\%2F\%2Fwww.fitchburgstate.edu\%2Foffices-services-directory\%2Ffitchburg-anti-violence-education\%2F\&sa=D\&sntz=1\&usg=AFQjCNFi5qy-wunMxX-hoWbA9YwT8aa4Ig}{Fitchburg
Anti-Violence Education (FAVE)}

FAVE collaborates with a number of community partners (e.g., YWCA
Domestic Violence Services, Pathways for Change) to meet our training
needs and to link survivors with community based resources. This site
also features
\href{http://www.google.com/url?q=http\%3A\%2F\%2Fwww.fitchburgstate.edu\%2Foffices-services-directory\%2Ffitchburg-anti-violence-education\%2Ffitchburg-anti-violence-education-resources\%2F\&sa=D\&sntz=1\&usg=AFQjCNF9KZ2O1AvPMLJTHdNg1DfmYYtgog}{resources}
for help or information about dating violence, domestic violence, sexual
assault and stalking. If you or someone you know is in an abusive
relationship or has been a victim of sexual assault, there are many
places to go for help. Many can be accessed 24 hours a day, seven days a
week, 365 days a year. On campus, free and confidential support is
provided at both Counseling Services and Health Services.

\emph{Community Food Pantry} Food insecurity is a growing issue and it
certainly can affect student learning. The ability to have access to
nutritious food is incredibly vital. The Falcon Bazaar, located in
Hammond G 15, is stocked with food, basic necessities, and can provide
meal swipes to support all Fitchburg State students experiencing food
insecurity for a day or a semester.

The university continues to partner with Our Father's House to support
student needs and access to food and services. All Fitchburg State
University students are welcome at the Our Father's House Community Food
Pantry. This Pantry is located at the Faith Christian Church at 40
Boutelle St., Fitchburg, MA and is open from 5-7pm. Each ``household''
may shop for nutritious food once per month by presenting a valid FSU
ID.

\begin{center}\rule{0.5\linewidth}{0.5pt}\end{center}

\hypertarget{academic-integrity}{%
\subsubsection{ACADEMIC INTEGRITY}\label{academic-integrity}}

The University ``Academic Integrity'' policy can be found online at
\href{http://www.fitchburgstate.edu/offices-services-directory/office-of-student-conduct-mediation-education/academic-integrity/}{http://
www.fitchburgstate.edu/offices-services-directory/office-of-student-conductmediation-education/academic-integrity/.}
Students are expected to do their own work. Plagiarism and cheating are
inexcusable. Any instance of plagiarism or cheating will automatically
result in a zero on the assignment and may be reported the Office of
Student and Academic Life at the discretion of the instructor.

Plagiarism includes, but is not limited to: - Using papers or work from
another class. - Using another student's paper or work from any class. -
Copying work or a paper from the Internet. - The egregious lack of
citing sources or documenting research.

\emph{If you're not clear on what is or is not plagiarism, ASK. The BEST
case scenario if caught is a zero on that assignment, and ignorance of
what does or does not count is not an excuse. That being said, I'm a
strong supporter of}
\emph{\href{https://en.wikipedia.org/wiki/Fair_Use}{Fair Use} doctrine.
Just attribute what you use--and, again, ASK if there's any doubt.}

\begin{center}\rule{0.5\linewidth}{0.5pt}\end{center}

\hypertarget{americans-with-disabilities-act-ada}{%
\subsubsection{AMERICANS WITH DISABILITIES ACT
(ADA)}\label{americans-with-disabilities-act-ada}}

If you need course adaptations or accommodations because of a
disability, if you have emergency medical information to share with the
instructor, or if you need special arrangements in case the building
must be evacuated, please inform the faculty member as soon as possible.

\begin{center}\rule{0.5\linewidth}{0.5pt}\end{center}

\hypertarget{technology}{%
\subsubsection{TECHNOLOGY}\label{technology}}

At some point during the semester you will likely have a problem with
technology. Your laptop will crash; your iPad battery will die; a
recording you make will disappear; you will accidentally delete a file;
the wireless will go down at a crucial time. These, however, are
inevitabilities of life, not emergences. Technology problems are not
excuses for unfinished or late work. Bad things may happen, but you can
protect yourself by doing the following:

\begin{itemize}
\item
  Plan ahead: A deadline is the last minute to turn in material. You can
  start---and finish---early, particularly if challenging resources are
  required, or you know it will be time consuming to finish this
  project.
\item
  Save work early and often: Think how much work you do in 10 minutes. I
  auto save every 2 minutes.
\item
  Make regular backups of files in a different location: Between Box,
  Google Drive, Dropbox, and iCloud, you have ample places to store and
  backup your materials. Use them.
\item
  Save drafts: When editing, set aside the original and work with a
  copy.
\item
  Practice safe computing: On your personal devices, install and use
  software to control viruses and malware.
\end{itemize}

\begin{center}\rule{0.5\linewidth}{0.5pt}\end{center}

\hypertarget{grading-policy}{%
\subsubsection{GRADING POLICY}\label{grading-policy}}

Grading for the course will follow the FSU grading policy below:

4.0: 95-100\\
3.7: 92-94\\
3.5: 89-91\\
3.3: 86-88\\
3.0: 83-85\\
2.7: 80-82\\
2.5: 77-79\\
2.3: 74-76\\
2.0: 71-73\\
0.0: \textless{} 70

\begin{center}\rule{0.5\linewidth}{0.5pt}\end{center}

\hypertarget{academic-resources}{%
\subsubsection{ACADEMIC RESOURCES}\label{academic-resources}}

\href{http://www.fitchburgstate.edu/offices-services-directory/tutor-center/writing-help/}{Writing
Center}

\href{http://catalog.fitchburgstate.edu/content.php?catoid=13\&navoid=851}{Academic
Policies}

\href{http://www.fitchburgstate.edu/offices-services-directory/disability-services/}{Disability
Services}

\href{https://www.getrave.com/login/fitchburgstate/}{Fitchburg State
Alert system for emergencies, snow closures/delays, and faculty
absences}

\href{http://www.fitchburgstate.edu/offices-services-directory/career-counseling-and-advising/careerservices/}{University
Career Services}

\begin{center}\rule{0.5\linewidth}{0.5pt}\end{center}



\end{document}
